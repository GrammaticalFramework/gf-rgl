%%%%%%%%%%%%%%%%%%%%%%%%%%%%%%%%%%%%%%%%%
% Beamer Presentation
% LaTeX Template
% Version 1.0 (10/11/12)
%
% This template has been downloaded from:
% http://www.LaTeXTemplates.com
%
% License:
% CC BY-NC-SA 3.0 (http://creativecommons.org/licenses/by-nc-sa/3.0/)
%
%%%%%%%%%%%%%%%%%%%%%%%%%%%%%%%%%%%%%%%%%

%----------------------------------------------------------------------------------------
%	PACKAGES AND THEMES
%----------------------------------------------------------------------------------------

%\documentclass[12pt]{beamer}
%\documentclass{beamer}
\documentclass[compress,table]{beamer} % compresses the navigation bar on top

\mode<presentation> {
% The Beamer class comes with a number of default slide themes
% which change the colors and layouts of slides. Below this is a list
% of all the themes, uncomment each in turn to see what they look like.

% best themes (ordered by preference)
\usetheme{Madrid} % has no navigation overview per slide
%\usetheme{Boadilla}% has no navigation overview per slide
%\usetheme{Warsaw} % navigation on top, title and author below
%\usetheme{PaloAlto} % nav at side, title and author also at side
%\usetheme{Berkeley}

%\usetheme{Szeged}
%\usetheme{Copenhagen}
%\usetheme{Luebeck}

%\usetheme{Malmoe}
%\usetheme{Bergen}
%\usetheme{CambridgeUS}
%\usetheme{AnnArbor}
%\usetheme{Marburg}

% not so nice themes
%\usetheme{default}
%\usetheme{Antibes}
%\usetheme{Berlin}
%\usetheme{Darmstadt}
%\usetheme{Dresden}
%\usetheme{Frankfurt}
%\usetheme{Goettingen}
%\usetheme{Hannover}
%\usetheme{Ilmenau}
%\usetheme{JuanLesPins}
%\usetheme{Montpellier}
%\usetheme{Pittsburgh}
%\usetheme{Rochester}
%\usetheme{Singapore}

% As well as themes, the Beamer class has a number of color themes
% for any slide theme. Uncomment each of these in turn to see how it
% changes the colors of your current slide theme.

\usecolortheme{dolphin}

%\usecolortheme{beaver}
%\usecolortheme{whale}
%\usecolortheme{lily}
%\usecolortheme{orchid}
%\usecolortheme{rose}
%\usecolortheme{seahorse}

%\usecolortheme{albatross}
%\usecolortheme{beetle}
%\usecolortheme{crane}
%\usecolortheme{dove}
%\usecolortheme{fly}
%\usecolortheme{seagull}
%\usecolortheme{wolverine}
}




%\setbeamertemplate{footline} % To remove the footer line in all slides uncomment this line
%\setbeamertemplate{footline}[page number] % To replace the footer line in all slides with a simple slide count uncomment this line

%\setbeamertemplate{navigation symbols}{} % To remove the navigation symbols from the bottom of all slides uncomment this line
\setbeamertemplate{headline}{} % removes the top bar
%\setbeamertemplate{enumerate items}[circle]
%\setbeamertemplate{enumerate items}[square]
\setbeamertemplate{enumerate items}[default]
\newcommand{\nologo}{\setbeamertemplate{logo}{}}

\usepackage{graphicx} % Allows including images
\usepackage{booktabs} % Allows the use of \toprule, \midrule and \bottomrule in tables
\usepackage{ulem} % needed for \sout (strike-out text)
\usepackage{hyperref} % can be used to create links to labels
\usepackage{appendixnumberbeamer} % use appendix, to avoid counting backup slides
\usepackage{multirow} % needed for \multirow{}{}{}

\usepackage{listings}

\lstdefinelanguage{gf}
{
  morekeywords={abstract, flags, cat, fun, incomplete, concrete, of, open, in, lincat, lin, resource, param, oper, variants, table, interface, instance, def, data, lindef, printname,},
  sensitive=false,
  morecomment=[l]{-{}-},
  morestring=[b]",
  stringstyle={\textit}
}
\lstset{language=gf,captionpos=b,numbers=none, numberstyle=\tiny, numbersep=5pt, basicstyle=\footnotesize\ttfamily, escapeinside={\%*}{*)},}

%----------------------------------------------------------------------------------------
%	TITLE PAGE
%----------------------------------------------------------------------------------------


% The short title appears at the bottom of every slide, the full title is only on the title page
\title[Latin grammar]{Implementation of a Latin Grammar in Grammatical Framework}

\author[Herbert Lange]{Herbert Lange\inst{1}}

\institute[University of Gothenburg]{
\begin{columns}
  \begin{column}{5.5cm}
    \centering
    \inst{1} Computer Science and Engineering\\University of Gothenburg and Chalmers University of Technology\\Gothenburg, Sweden
    \textit{herbert.lange@cse.gu.se}
  \end{column} 
\end{columns}
}

\date[DATeCH 2017]{DATeCH 2017, Göttingen, June 1st-2nd} % CPSS

\logo{\includegraphics[width=0.5\textwidth]{LO_CH-GU.eps}}
\begin{document}

\begin{frame}
  \titlepage % Print the title page as the first slide
\end{frame}

\begin{frame}{Overview} % Table of contents slide, comment this block out to remove it
%% Throughout your presentation, if you choose to use \section{} and \subsection{} commands, these will automatically be printed on this slide as an overview of your presentation
\tableofcontents%[hideallsubsections]%[pausesections]
\end{frame}

\section{Background}
\subsection{Latin}

\begin{frame}{Latin}
  \begin{itemize}
  \item Indo-European language
  \item development almost from 240 b.c. to the beginning
    of the 20th century (and in some fields still continues)
  \item Usual focus on classic period (from the first public speeches of M. Tullius Cicero (ca.
    80 b.c. to ca. 117 a.d.)
  \item Strong inflectional and synthetic language
  \item Rather free word order (but strong tendencies within certain text domains and time periods, also not all combinations seem to be acceptable)
  \end{itemize}
\end{frame}

%% {\nologo
%%   \begin{frame}
%%     \includegraphics[width=\textwidth]{Life-of-Brian-Latin-2}
%%   \end{frame}
%% }

\subsection{Grammatical Framework}
\begin{frame}{Grammatical Framework}
  \begin{itemize}
  \item Modern grammar formalism based on type theory and inspired by functional programming languages (especially Haskell)
  \item Variant of context-free grammars extended by so called tables and records
  \item Expressivity equivalent to Parallel Multiple Context-Free Grammars (PMCFG) (mildly context-sensitive)
  \item Separation in Abstract and Concrete Syntax
  \item Open Source
  \end{itemize}
\end{frame}
\begin{frame}[fragile]{Example}
  \begin{lstlisting}
param Number = Sg | Pl ;
param Case = Nom | Acc | Dat | Abl | Voc ;
param Gen = Masc | Fem | Neutr ;

lincat N = { s : Number => Case => Str ; g : Gender } ;

lin
  man_N = {
    s = table {
      Sg => table {
         Nom => "vir" ; Acc => "virum " ; Gen => "viri" ;
         Dat => "viro" ; Abl => "viro" ; Voc => "vir" } ;
      Pl => table {
         Nom => "viri" ; Acc => "viros" ; Gen => "virorum" ;
         Dat => "viris" ; Abl => "viris" ; Voc => "viri" ; }
  } ;
  g = Masc
}
  \end{lstlisting}
  % Here I should come up with a minimal example of a grammar with tables and records
\end{frame}

\subsection{Resource Grammar Library}

\begin{frame}{Resource Grammar Library}
  \begin{itemize}
  \item Grammarians equivalent to a Programming Languages standard library
  \item Common basis for multilingual applications (and machine translation)
  \item Also Open Source, and this grammar is part of it
  \end{itemize}
\end{frame}

\section{Grammar}

\begin{frame}{General Concept}
  \begin{itemize}
  \item Originally: Translate information from a standard (school) grammar book into a computerized form
  \item Implement the RGL abstract syntax
  \item Now: Application-specific resources especially for language learning
  \item Constituent grammar not dependency grammar (not as cool but conversion between GF trees and UD is possible)
  \end{itemize}
\end{frame}
\subsection{Lexicon}
\begin{frame}{Lexicon}
  \begin{itemize}
  \item Basic Lexicon with ca 350 entries
  \item Contains mostly base forms, uses morphological rules to generate the whole paradigms
  \item Main problem: modern concepts (e.g. refrigerator) and homonyms (e.g. bank)
  \item Use of web-based and collaborative resources (e.g. Latin Wikipedia, Wiktionary)
  \item Work in progress: Adopt other lexical resources
  \end{itemize}
\end{frame}
\subsection{Morphology}
\begin{frame}{Morphology}
  \begin{itemize}
  \item Strongly inflectional but quite regular morphology
  \item Use as little information as possible to generate the whole paradigm (smart paradigms)
  \item Several inflection classes for different lexical categories
  \item Use tables for word forms depending on grammatical features
  \end{itemize}
\end{frame}
\begin{frame}[fragile]
\begin{table}
  \resizebox{\textwidth}{!}{
        \begin{tabular}{l|l|l|l}
          Word class & Inherent & Parametric & No. of Inflection classes\\
          \hline
          Noun & Gender & Number, Case & 5\\
          Adjective & & Degree, Gender, Number, Case & 3\\
          Verb (active) & & Anteriority, Tense, Number, Person & 4 regular, 4 deponent \\
          Determiner & Number & Gender, Case \\
      \end{tabular}}
    \end{table}
     \begin{table}[h]
      {\small
        \begin{tabular}{l|l}
          Feature & Values \\
          \hline
          Gender & Feminine, Masculine, Neuter \\
          Number & Singular, Plural \\
          Case & Nominative, Genitive, Dative, Accusative, Ablative, Vocative\\
          Degree & Positive, Comparative, Superlative \\
          Anteriority & Anterior, Simultaneous \\
          Tense & Present Indicative, Present Subjunctive, Imperfect Indicative, \\
          & Imperfect Subjunctive, Future Indicative, Future Subjunctive\\
          Person & 1, 2, 3 \\
      \end{tabular}}
    \end{table}
\end{frame}
\subsection{Syntax}
\begin{frame}{Syntax}
  \begin{itemize}
  \item Syntax rules use basic and smaller parts to assemble larger parts up to the sentence level
  \item Challenge: Free word order
  \item Decision what parts can be completely assembled at what point (and what has to be kept apart)
  \item Use records to keep parts of a phrase separate
  \item Postpone decision on word order as long as possible
  \end{itemize}
\end{frame}

\section{Results}

\begin{frame}{Results}
  \begin{itemize}
  \item Comprehensive Latin morphology
  \item Implemented about 1/3 of the RGL syntactic functions (but some of them are rather obscure)
  \item Already usable in applications
  \item Future: Adding rules as they are needed and large-scale evaluation
  \end{itemize}
\end{frame}
\section{Applications}

\begin{frame}{Applications}
  \begin{itemize}
  \item Our main focus: Language learning\\
    Grammar-based and gamified computer-aided language learning application for beginner's level
  \item Other possible applications in Digital Humanities\\
    Giving access to cultural heritage e.g. with a translation app for epigraphs
  \end{itemize}
\end{frame}

\begin{frame}
  {\huge
  Thanks for your attention \\[3em]
  \pause
  Questions? \\[2em]
  }
  Source:\\ \url{wwww.grammaticalframework.org} and \url{https://github.com/daherb/GF-latin}
\end{frame}
% backup slides
\appendix

%% {\nologo 
%%   \begin{frame}
%%     \begin{center} \includegraphics[scale=0.5]{delirant-isti-romani} \end{center}
%% \end{frame}
%% }


{\nologo 
  \begin{frame}[fragile]
    \begin{lstlisting}[basicstyle=\tiny\ttfamily]
noun : Str -> Noun = \lexform -> 
case lexform of {
- - noun1, noun2us/um/er, noun4 and noun5 are the functions for the different
- - declension classes. The 2nd class is split in three subclasses
_ + "a"  => noun1 lexform ;
_ + "us" => noun2us lexform ;
_ + "um" => noun2um lexform ;
- - "Predef.tk n word" removes a suffix of length n from word
_ + ( "er" | "ir" ) => noun2er lexform ( (Predef.tk 2 lexform) + "ri" ) ;
_ + "u"  => noun4u lexform ;
_ + "es" => noun5 lexform ;
- - Predef.error stops with a given error message
_  => Predef.error ("3rd declinsion cannot be applied " ++
      "to just one noun form " ++ lexform)
} ;
  \end{lstlisting}
\end{frame}
}

{\nologo 
  \begin{frame}[fragile]
    \begin{lstlisting}[basicstyle=\tiny\ttfamily]
mkClause : NounPhrase -> VerbPhrase -> Clause = \np,vp -> {
  s = \\tense,anter,pol,vqf,order => case order of {
    SVO => np.s ! Nom ++ negation pol ++ vp.compl ! Ag np.g np.n Nom ++ vp.s ! VAct ( anteriorityToVAnter anter ) ( tenseToVTense tense ) np.n np.p ! vqf ++ vp.obj ;
    VSO => negation pol ++ vp.compl ! Ag np.g np.n Nom ++ vp.s ! VAct ( anteriorityToVAnter anter ) ( tenseToVTense tense ) np.n np.p ! vqf ++ np.s ! Nom ++ vp.obj ;
    VOS => negation pol ++ vp.compl ! Ag np.g np.n Nom ++ vp.s ! VAct ( anteriorityToVAnter anter ) ( tenseToVTense tense ) np.n np.p ! vqf ++ vp.obj ++ np.s ! Nom ;
    OSV => vp.obj ++ np.s ! Nom ++ negation pol ++ vp.compl ! Ag np.g np.n Nom ++ vp.s ! VAct ( anteriorityToVAnter anter ) ( tenseToVTense tense ) np.n np.p ! vqf ;
    OVS => vp.obj ++ negation pol ++ vp.compl ! Ag np.g np.n Nom ++ vp.s ! VAct ( anteriorityToVAnter anter ) ( tenseToVTense tense ) np.n np.p ! vqf ++ np.s ! Nom ;
    SOV => np.s ! Nom ++ vp.obj ++ negation pol ++ vp.compl ! Ag np.g np.n Nom ++ vp.s ! VAct ( anteriorityToVAnter anter ) ( tenseToVTense tense ) np.n np.p ! vqf 
  } 
} ;
\end{lstlisting}
\end{frame}
}
\end{document} 




